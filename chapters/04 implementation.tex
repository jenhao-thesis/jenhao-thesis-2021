\chapter{Implementation} 
\label{chapter:implementation}
This chapter describes the design of smart contracts and provides the detailed implementation of each function that is adopted in this paper. The smart contract diagram as shown in Figure~\ref{fig:smart_contract_diagram},  all of the organizations have common \(OMgr\) for managing users' information and storing users' status. The ecosystem initiator enables blockchain-based functionality by deploying a \(OMgr\). It offers application binary interface (ABI) files for all participants to easily call smart contract functions.

\begin{figure}[hb]
    \centering
    \includegraphics[height=!,width=1\linewidth,keepaspectratio=true]{figures/smart_contract_diagram.png}
    \caption{{\footnotesize Smart Contract Diagram}}
    \label{fig:smart_contract_diagram}
\end{figure}
\section{Smart contract design}
\subsection*{Organization Manager}
The \(OMgr\) is used to manage information that is related to users and organizations, i.e., it records data attributes, users' identity hash value. In Table~\ref{table:userinfo}, the \textit{UserInfo} structure represents an independent \(DI\). After the identity verification process is done as shown in Figure~\ref{fig:identityVerification}, the \(DI\) will be created for users automatically.

\input{tables/table-userinfo}

This smart contract provides several functions to create user identity, bind account with Ethereum address, and create exclusive \(ACMgr\) for the user. And it also defines events in order that the smart contract can emit events to record logs on the block. That enables traceability of identity creation processes. The most important functions we used in \(OMgr\) are \textit{addUser}, \textit{bindAccount} and only the legitimate organizations are able to trigger these functions.\par 
The function \textit{addUser} is invoked after finishing identity verification, the \(OMgr\) check whether the \(ID\) exist on the blockchain. Then if the \(ID\) is new one, the \(OMgr\) will create \textit{UserInfo} for users. Otherwise, the \(OMgr\) will append the address of \(Org\) who triggers this function to the variable \textit{orgs} in \textit{UserInfo}.

\begin{figure}[hb]
    \centering
    \includegraphics[height=!,width=0.8\linewidth,keepaspectratio=true]{figures/smart_contract_deployment.png}
    \caption{{\footnotesize The relation between smart contracts}}
    \label{fig:smart_contract_deployment}
\end{figure}

\subsection*{Access Manager}
Each user manages access control list of data.
// data structure

\section{Third Party Login}
\section{Integration Account}
\section{Data Sharing}
\section{Token design}
// jwt format(including iss, sub, hashed)
\section{Data privacy protection}
// which data will be save in BC and in DB.