\chapter{Introduction}
\label{chapter:intro}
\section{Motivation}
In recent years social login services have become ever more prevalent such as Google, Facebook, or Twitter. The user can sign into a third party website without creating a new account. Those services provide unified identity management through social login and allow the user to confirm the access permission. Although social login seems to have a lot of benefits, e.g., security, convenience, and ease of use, there has a concern about those service providers may collect sensitive information about the user~\cite{gafni2014social}.
Besides, those social login services are mainly based on the centralized systems that enforce data permit across different parties. If one third party application sends a retrieval request for the data, it must obtain permission from the centralized party.\par
Open banking is a hot topic in the financial industry nowadays. It aims to share customer's financial data with different organizations and required consent. The banking institutions disclose APIs to third-party service providers for creating new services, analytics, financial products to improve customer experience. It is not only meeting customer needs but also help third-party service provider to create innovate activity for discover prospective customers and accelerate financial inclusion. In recent years, open banking has been adopted in various stages in countries around the world. Three phases have been defined by Open Banking and each stage represents sharing data scope. In Taiwan, Financial Supervisory Commission (FSC) has approved several banks allowing to join in the second phase of Open Bnaking~\cite{thepaypers_2021}.\par

\begin{itemize}
    \item Public information: interest rates, exchange rates, mortgage rates, foreign currency exchange rates, etc.
    \item Customer data: accounts, credit cards, personal information, etc.
    \item Transaction information: TSPs need to obtain customer's permission to access their data and to enable transactions and payment.
\end{itemize}\par

A key issue is the privacy of customer's personal data including customer's deposits, loans, investments, and account information. When financial institutions disclose APIs to TSPs, the system has many critical concerns such as malicious attacks, tampering. Once the attacker hacks the system, the customer data will be exposed and may cause enormous losses. Blockchain technology can provide data storage, access control, transaction security, and tamper-proof data so that it can protect customer data privacy~\cite{wang2020blockchain}.\par
Blockchain technology brings numerous benefits in a variety of industries, providing more security in trustless environments. The blockchain is a distributed digital ledger that storing records or data in blocks and those blocks are linked through cryptographic proofs so that the attacker can not temper any data. In Ethereum blockchain, it provides smart contract to us to deploy autonomous applications without third party and interact with smart contract on Ethereum network.\par

\section{Objective}

In order to address these problems, we propose blockchain-based identification and access control system for Open Banking ecosystem in this thesis. The system allows organization administrators to interact with blockchain for register the digital identity of the customer by the actual identity, and customer also can manage their digital identity and control their data access by calling smart contract functions directly. \par