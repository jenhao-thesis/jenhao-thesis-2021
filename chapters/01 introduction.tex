\chapter{Introduction}
\label{chapter:intro}
\section{Motivation}
In recent years social login services have become ever more prevalent such as Google, Facebook, or Twitter. The user can sign into a third party website without creating a new account. Those services provide unified identity management through social login and allow the user to confirm the access permission. Although social login seems to have a lot of benefits, e.g., security, convenience, and ease of use, there has a concern about those service providers may collect sensitive information about the user~\cite{gafni2014social}.
Besides, those social login services are mainly based on the centralized systems that enforce data permit across different parties. If one third party application sends a retrieval request for the data, it must obtain permission from the centralized party.\par
Open Banking is a hot topic in the financial industry nowadays. It aims to share customer's financial data with different organizations and required consent. The banking institutions disclose APIs to third-party service providers for creating new services, analytics, financial products to improve customer experience. It is not only meeting customer needs but also help third-party service provider to create innovate activity for exploring prospective customers and accelerate financial inclusion. In recent years, Open Banking has been adopted in various stages in countries around the world. Three phases have been defined by Open Banking and each stage represents sharing data scope. In Taiwan, Financial Supervisory Commission (FSC) has approved several banks allowing to join in the second phase of Open Bnaking~\cite{thepaypers_2021}.\par

\begin{itemize}[noitemsep]
    \item \textbf{Phase 1 Public information:} interest rates, exchange rates, mortgage rates, foreign currency exchange rates, etc.
    \item \textbf{Phase 2 Customer data:} accounts information, loan, deposit, credit cards, personal information, etc.
    \item \textbf{Phase 3 Transaction information:} account integration, payment, debit authorization, settlement of the loan, etc.
\end{itemize}\par

A key issue is the privacy of customer's personal data including customer's deposits, loans, investments, and account information. When financial institutions disclose APIs to TSPs, the system has numerous critical concerns such as malicious attacks, tampering. Once the attacker hacks the system, the customer data will be exposed and may cause enormous losses. Blockchain technology can provide data storage, access control, transaction security, and tamper-proof data so that it can protect customer data privacy~\cite{wang2020blockchain}.\par
Blockchain technology brings numerous benefits in a variety of industries, providing more security in trustless environments. The blockchain is a distributed digital ledger that storing records or data in blocks and those blocks are linked through cryptographic proofs so that the attacker can not temper any data. In Ethereum blockchain, it provides smart contract to us to deploy autonomous applications without third party and interact with smart contract on Ethereum network.\par

\section{Objective}

In order to address these problems, we propose blockchain-based identification and access control system for Open Banking ecosystem in this thesis. The system allows organization administrators to interact with blockchain for register the digital identity of the customer by the actual identity, and customer also can manage their digital identity and control their data access by calling smart contract functions directly. If any organization or financial institution wants to participate in existed blockchain-based Open Banking ecosystem, the system platform will create an Ethereum account for them so that they can prove they have ownership for Ethereum address. Customer registers their bank account and then bind their actual identity to their Ethereum address for enabling blockchain-based third-party authentication. After binding, customers have a unique digital identity on blockchain, and they can log in to another financial institution or third-party service provider without filling any form. In addition to binding, customers can also integrate existed accounts into their unique digital identity on blockchain.\par
With blockchain technology, the system is more secure, traceable, transparent, and tamper-resistant. TSP also does not have to create authentication systems and it can ensure customer's digital identity doesn't have tampered with. TSP is required to invoke smart contract function to confirm their access permission and access scope. After the financial institution successfully authenticating the TSP, it issues access tokens to TSP. TSP can request customer data with access token through API.\par
This thesis is organized as follows nine chapters. Chapter 1 and 2 introduce the background knowledge of blockchain and related work in Open Banking ecosystem. Chapter 3 describes the system overview, its scenario and explains how it work. Chpater 4 presents the smart contract used in our system and its workflow. Chapter 5 illustrates case studies and experimental validation. Chapter 6 presents demonstration of our system. Chapter 7 describes the evaluation of our system performance, including gas consumption and its throughput. Finally, we summarize our system functions, our finding, and discussion in Chapter 8 and 9.\par