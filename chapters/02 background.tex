\chapter{Background}
\label{chapter:background}

\section{Ethereum}
    \subsection{Smart Contract}
        Ethereum~\cite{buterin2014next} was proposed in 2014 by Vitalik Buterin, it is a decentralized, open-source blockchain, and it also supports smart contracts. Because Ethereum enables smart contracts, the programmers can build their distributed applications by writing smart contract programs, e.g., Solidity. With smart contract technology, they can build self-enforce, self-verify and tamper-proof systems, such as voting systems, healthcare, supply chain, financial service, and so on. \par        
        In the Etheruem platform, the block not only stores transaction records, but also stores smart contract so that Ethereum has the capability to compute business logic. The blockchain uses Merkle tree to store the transactions in every block.\par

    \subsection{Merkle tree}
    \begin{figure}[hb]
        \centering
        \includegraphics[height=!,width=1\linewidth,keepaspectratio=true]{figures/merkle_tree_in_BC.png}
        \caption{{\footnotesize Merkle tree in Bitcoin blockchain}}
        \label{fig:merkleTreeInBC}
    \end{figure}
    Merkle trees are important in blockchain technology. A Merkle tree is a tree in which every node is stored with the hash of data. Each node is the hash of his leaves. In Bitcoin blockchain, it uses the Merkle tree as proof to make sure the data block can't be tampered with, it is not possible to modify the data after the data written in the blockchain as shown Figure~\ref{fig:merkleTreeInBC}.\par
    \begin{figure}[htb]
        \centering
        \includegraphics[height=!,width=1\linewidth,keepaspectratio=true]{figures/merkle_tree_in_Eth.png}
        \caption{{\footnotesize Merkle tree in Ethereum}}
        \label{fig:merkleTreeInEth}
    \end{figure}
    The Merkle tree in Ethereum is as Figure~\ref{fig:merkleTreeInEth}. Every block header has not only one tree, but three trees for transactions, receipts, and state. The state root is the root hash of the Merkle Patricia tree that used to store the entire state of the Ethereum blockchain, like account balances, contract storage, and contract code. Unlike Bitcoin blockchain, Ethereum uses Merkle Patricia tree as the state tree, it consists of a map struct, the keys are account address and the values are account declarations such as the balance, nonce.

    \subsection{Elliptic Curve Cryptography (ECC)}
        Elliptic Curve Cryptography is public key cryptography based on elliptic curves. In Ethereum, they use ECC to generate the key pair. Initially, the base point \(G\) on elliptic curve \(CURVE\) must be provided, and then randomly generate 256-bits integer \(m\) as a private key which is used to sign message or transaction. If given \(m\), it is easy and fast to find 512-bits public key \(P\). But if given \(P\), it is impossible to find \(m\) because of Elliptic Curve Logarithm Problem. The Ethereum address is the last 20-bytes of  Keccak-256 hash of the public key.\par

        \begin{equation}
            P=[m]G
        \end{equation}

        Elliptic Curve Digital Signature Algorithm (ECDSA) is used to create a digital signature of data that uses elliptic curve cryptography. ECDSA isn't used to encrypt any data, it makes sure that the data was not tampered with. In the Ethereum blockchain, it used to prove ownership of an address without revealing private key. \par
        After signing message using ECDSA through private key, the digital signature consists of three values: \(r, s, v\). \(r\) and \(s\) are the values used in standard ECDSA signatures, and \(v\) means recover id that used to recover signed message. In blockchain (e.g., Bitcoin, Ethereum), we called it public key recovery. If we get \((r, s)\) and message, we can compute the public key to verify message.\par
    
    \newpage
    \section{MetaMask}
        \begin{figure}[htb]
            \centering
            \includegraphics[height=!,width=1\linewidth,keepaspectratio=true]{figures/architecture_of_dapp.png}
            \caption{{\footnotesize Architecture of DApp}}
            \label{fig:architecture_of_dapp}
        \end{figure}
        MetaMask~\cite{metamask} is the most popular crypto wallet for accessing Ethereum distributed application (DApp). This tool can enable web3 API in website so that users can interact with various Etehereum blockchain from Javascript~\cite{web3.js}, e.g., Mainnet, Testnet. It also creates accounts by the user themself. The user of MetaMask can create and manage their accounts; moreover, MetaMask provides an interface that user can perform a transaction to the connected blockchain.\par
        Because the user securely manages owned Ethereum account through MetaMask, the user can use their private key to sign a transaction or sign data to prove ownership of an account. Using this crypto wallet, the developers of decentralized applications can build their own cryptocurrencies and focus on designing and implementing functions of smart contracts. There are several different architectures for distributed apps~\cite{wessling2018engineering}, the most common way is the developer design frontend that can allow the users to interact with the business logic by MetaMask. Or, the user can send a transaction to blockchain directly.\par
        In the MetaMask wallet, it seems to have a lot of benefits. Firstly, the keys are stored in the user's browser and it doesn't store on wallet provider's server, so the user can manage his/her private key and public key without server. Secondly, it provides an easy to use interface, every user can send and receive cryptocurrency or token.\par
        Regarding the architecture of Dapp, Figure~\ref{fig:architecture_of_dapp} shows that the web3.js libraries can enable user's browser to interact with blockchain so that users can read and write data from smart contracts, send transactions between accounts. 



\newpage

\section{Open Banking}
- Open Banking flow
- Current TSP in Taiwan
- Three stage table
\subsection*{Third-party Service Provider (TSP)}


\newpage

\section{Related works}
In this section, we will present an overview of the related literature on blockchain-based access control, identity management, data sharing and the Open Banking ecosystem. We don't just point out the differences between previous studies and our research, but we also look at the tradeoff between the access control models.
\subsection{Identity integration}
Mudliar \emph{et al.}~\cite{mudliar2018comprehensive} proposed an integration solution of national identity using blockchain technology. Each entity (regime officials and citizens) has a unique blockchain address for identification and integrates it with a unique identity number issued by the government. As the authors write: "With the possibilities of the blockchain technology, all such identities can be consolidated and only one identification can be used for the endless applications".

\subsection{Access control}
In traditional access control management, the most common solution is PKIs, but it has some concerns about scalability and granularity. Paillisse \emph{et al.}~\cite{paillisse2019distributed} presented a blockchain-based approach to address these problems. They took advantage of blockchain to record and distribute access control policies. Firstly, they selected Locator/ID Separation Protocol (LISP) to perform access control and modified this protocol for communication with the blockchain. Secondly, an open-source blockchain, the Hyperledger Fabric, was used to refresh access policies in the control plane. Thirdly, they constructed a prototype system and used Group-Based Policy (GBP) for network administrators to specify network configuration.\par

Daraghmi \emph{et al.}~\cite{daraghmi2019medchain,daraghmi2019unichain} described a blockchain-based system for electronic medical records and academic records, using blockchain smart contracts to provide secure and efficient data access while protecting the patients' privacy. These systems enabled the data owner to securely control accesses their records.\par

Fu \emph{et al.}~\cite{fu2020soteria} proposed a user rights management system that aims to protect user privacy through enforcing executable sharing agreement. They also adopted multi-layer blockchain architecture to satisfy CAP (consistency, availability, and partition tolerance) theorem.\par

Rouhani \emph{et al.}~\cite{rouhani2020distributed} proposed a distributed Attributed-Base Access Control (ABAC) system that can provide auditing of access attempts. This work has focused on addressing audibility and scalability. They also applied the blockchain-based solution to the digital library for validation.\par

Guo \emph{et al.}~\cite{guo2019multi} introduced the smart contract technology for multi-authority ABAC. In their proposed architecture, the interactions among data users, data owners, and authorities are built on the blockchain smart contract.\par

In many cases, users must be both authenticated and authorized prior to entering the system. Traditional access control models have resolved the issues. For example, Attributed-Base Access Control (ABAC) utilizes a set of attributes (e.g., user attributes, environmental attributes, and resource attributes) to set permissions; Role-Based Access Control (RBAC) granted access permissions depending on users' role in the organization. Generally, ABAC is a flexible and fine-grained mechanism, but it is more complex and requires well-designed management.\par

In view of all the access control mechanisms mentioned so far, the focus of our research is users' consent and secure authentication and authorization. In the current research, we investigate blockchain technology, access control mechanisms. The challenge arises when users have multiple digital identities and need to monitor the use of their personal information by another organization.\par

\subsection{Open banking}
Open banking brings not only opportunities but also challenges all over the world, such as malicious attacks, identity theft, data leakage, and privacy disclosure, etc. Fortunately, blockchain, with the properties of tampering resistance, transparency, and verifiability, perfectly matches the requirements of open banking services. We will review four types of blockchain research in the open banking area with different aspects; they are data privacy protection, access with customers consent, provenance data for tracking TSPs accessing behavior, and open API management.\par

First, Hao wang et al.~\cite{wang2020blockchain} proposes a data privacy classification method and disclosure scheme for achieving customer privacy protection. Their proposed idea works well in dealing with deficiencies when implementing blockchain in open banking, such as the granularity of privacy-preserving, various financial data, the complexity of banking subsystems, and hierarchical data management. Second, Mukhopadhyay et al.~\cite{mukhopadhyay2021blockchain} systematically illustrate the importance of customer consent management. This study also proposes a TSP rating design for customer's reference before making consent. Third, Zhiyu Xu et al.~\cite{xu2020ppm} indicate that sensitive financial data requires higher authentication and provenance for participants. Thus they develop a provenance-provided data-sharing model for customers and banks to audit and track the TSPs' historical access behavior. Fourth, Q. Zhang et al.~\cite{zhang2019obbc} discuss the security risks when financial institutes provide application programming interfaces (APIs) to TSPs. And they conceptualize a data-sharing scheme and an API consensus mechanism that aims to prevent the open APIs from being malicious forged.
