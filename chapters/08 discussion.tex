\chapter{Discussion}
\label{chapter:discussion}

In this chapter, we will first discuss the characteristics and related issues of our proposed framework. Then we will describe how applications provided by different industries apply this framework to enable blockchain functionality.
\par 

Firstly, the users and organization administrators can create Ethereum accounts as digital identities by using crypto wallets or tools. The private key of this account can be used in any blockchain network to initiate transactions or triggered contracts, but balances and smart contracts are independent in different chains. Secondly, each organization provides a DApp that improves the user experience. For example, users can access their financial data from different banks through a unique digital identity stored on the blockchain without remembering bank accounts and passwords. Thirdly, since our system focuses on user consent, any data access must obtain permission from the user. We implemented user consent by restricting that only the owner of this \textit{ACMgr} could call functions. Therefore, banks and TSPs did not interfere with deciding of the user.
\par 

Our decentralized access control systems bring more opportunities and challenges: (1) \textbf{Innovative service}: This framework provides organizations with the ability to improve current services. We can extend existing products and services and develop new products through our data sharing model and identity integration. Therefore, more traditional banks and Fintech companies are willing to join this ecosystem to provide more convenient services. (2) \textbf{Reducing the risk}: Due to the characteristics of blockchain technology, we can trust the validity of digital identity. It cannot be corrupted and promotes transparency and tamper-proof. Unique user identity and data sharing between banks facilitates KYC (Know Your Customer) procedures and helps AML (Anti-Money Laundering).

