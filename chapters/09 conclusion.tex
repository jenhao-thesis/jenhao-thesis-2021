\chapter{Conclusion}
\label{chapter:conclusion}

Our research was motivated by identity authentication and data sharing, that open banking needed a secure manner to decentralized access right and provided user data API under protecting uses' privacy. In this research, we presented the design of an access control framework based on blockchain smart contracts. In order to perform self-sovereign identity, we presented identity integration that can be used for participating in this ecosystem, as well as controlling data access rights. We described the system architecture, workflow of each functionality, and interactions among users, organizations, and TSPs. Ethereum smart contract can distribute data access rights to the whole network, provide tamper-proof identity, and audit log. 
The prototype of our proposed system was implemented to validate our design on a private network of multiple Geth nodes. Our proposed system solved the shortcomings of the conventional centralized system and enabled distributed access control functionality. In this framework, users can authorize/revoke data access right directly without any organizations. The organization allows external data access requests from legitimate TSPs if TSPs obtain the consent of the user. 
\par

We apply public key cryptography (ECC) and digital signature algorithm (ECDSA) to DApp for user authentication and compare it with the general system (username and password). Although the general system has better performance, the blockchain based login provides a more secure and convenient way, i.e., account integration, decentralization system.\par

Our experimental evaluation shows that our system can handle at least 5,000 user operations (identity creation, login requests, binding requests, and access right authorization/revocation) in a certain response time and can compatible with general systems.\par

In this research, we focused on the design of user authentication and data sharing for construct a decentralized system. However, we didn't consider user data synchronization in different organizations. If the user updates the user profile in an organization, other organizations won't update simultaneously. It is important to think of whether the user's data is stored in the blockchain.