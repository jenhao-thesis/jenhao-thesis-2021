\begin{abstractzh}

    隨著社群平臺愈來愈普及,多數網站提供第三方登入(social login),讓初次使用的用戶可以用第三方平臺現有的帳號完成註冊及登入,
    幫助用戶免於記下不同網站的帳號及密碼,亦不用填寫繁雜的註冊表單。對於用戶而言,可以達到更好的用戶使用體驗;對於應用程式開發人員而言,不必自行管理個資、建立會員系統,由第三方平臺負責管理,而當需要在提供多種不同的服務時,可以使這些服務支援同一種第三方平臺驗證方式,即可達到單一登入(SSO)功能。但第三方登入系統仍屬於中心化系統,用戶的數位身分及資料皆屬於第三方平臺,因此用戶在使用服務的同時,也提供個人資料給服務提供者。\par
    然而,以太坊區塊鏈擁有防偽造、防竄改及去中心化的特性,可以安全、有效存放紀錄於鏈上,達到透明性且安全性。透過以太坊虛擬機及以太坊智能合約可以建構去中心化的應用程式,提供更完整、多樣的功能。\par
    本篇論文提出使用以太坊區塊鏈技術應用於開放銀行(Open Banking),使銀行機構與第三方服務業者(TSP)以安全、透明方式合作,透過區塊鏈技術管理並整合用戶身分,方便用戶管理及存取資料,且提供第三方登入的優點,並使得去中心化平臺成為信任的第三方,負責管理對應用戶身分並且提供存取控制之功能,同時兼具區鏈特性,提升安全性。除了驗證用戶身分功能外,亦提供用戶授權功能,用戶可以自行決定授權範圍及對象,授權對象包含第三方服務(TSP)業者。

\end{abstractzh}

\begin{flushleft}
    \textbf{關鍵字}: 區塊鏈、智能合約、第三方服務提供者、開放銀行
\end{flushleft}